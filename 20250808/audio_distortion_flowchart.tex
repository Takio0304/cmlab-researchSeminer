\documentclass[dvipdfmx]{standalone}

\usepackage{tikz}
\usetikzlibrary{shapes.geometric, arrows, positioning}

% フローチャートの各要素のスタイルを定義します
% processスタイル: 通常の四角い箱
\tikzstyle{process} = [
    rectangle, 
    draw=black, 
    thick, 
    text centered, 
    minimum height=1.2cm, % 箱の最小の高さ
    text width=5cm  % テキストがこの幅を超えたら自動で改行
]

% lineスタイル: ノード間をつなぐ実線
\tikzstyle{line} = [thick]

\begin{document}

% tikzpicture環境内で図を描画します
% node distanceでノード間の基本的な距離を指定します (縦 and 横)
\begin{tikzpicture}[node distance=2cm and 2.5cm]

% --- ノードの配置 ---

% Level 0 (一番左の階層)
\node (root) [process, text width=2cm] {音割れの原因};

% Level 1 (中央の階層)
% yshiftで垂直方向の位置を調整し、分岐間のスペースを確保します
\node (meta)  [process, right=of root, yshift=4.5cm] {MetaQuest3のスピーカーの問題};
\node (unity) [process, right=of root, yshift=-3.0cm] {Unityの問題};

% Level 2 (一番右の階層)
\node (meta1) [process, right=of meta, yshift=1.2cm] {スピーカーのボリュームを下げる};
\node (meta2) [process, right=of meta, yshift=-1.2cm] {PCから音を同時出力する};

% 4つの項目を均等に配置するためにyshiftを調整
\node (unity1) [process, right=of unity, yshift=3.6cm] {音源の音響パワーレベルを下げる};
\node (unity2) [process, right=of unity, yshift=1.2cm] {AudioMixerを使わずに音源合成};
\node (unity3) [process, right=of unity, yshift=-1.2cm] {可聴化のみを別のソフトで実行する};
\node (unity4) [process, right=of unity, yshift=-3.6cm] {Unityのバージョンアップ};


% --- ノード間の接続 ---

% rootからの接続
% (node.east)はノードの右辺中央のアンカーポイントを示します
% -- ++(1.25cm,0) で右に1.25cm水平線を描画します
% |- (node.west) でそこから垂直に曲がり、ターゲットノードの左辺中央に接続します
\draw[line] (root.east) -- ++(1.25cm,0) |- (meta.west);
\draw[line] (root.east) -- ++(1.25cm,0) |- (unity.west);

% metaからの接続
\draw[line] (meta.east) -- ++(1.25cm,0) |- (meta1.west);
\draw[line] (meta.east) -- ++(1.25cm,0) |- (meta2.west);

% unityからの接続
\draw[line] (unity.east) -- ++(1.25cm,0) |- (unity1.west);
\draw[line] (unity.east) -- ++(1.25cm,0) |- (unity2.west);
\draw[line] (unity.east) -- ++(1.25cm,0) |- (unity3.west);
\draw[line] (unity.east) -- ++(1.25cm,0) |- (unity4.west);

\end{tikzpicture}

\end{document}

%====================================================================================
% LaTeX Document Class and Packages (pLaTeX version)
% 文書クラスとパッケージの読み込み (pLaTeX対応版)
%====================================================================================
\documentclass[10pt]{jsarticle}
\usepackage{cmlab_4.00}				%% 研究室や学会指定のパッケージ
\usepackage{cmlabplus}				%% 拡張パッケージ
\usepackage[dvipdfm]{graphicx}	    %% 図の貼り付け用パッケージ
\usepackage{url}
\usepackage{balance}
\usepackage{subcaption}
\captionsetup[sub]{labelfont=sf}
\captionsetup{labelsep=quad}
\pagestyle{cmlabw}

% 必要に応じて追加のパッケージを読み込みます
% \usepackage{amsmath} % 高度な数式用

%====================================================================================
% Document Start
% ドキュメント本体の開始
%====================================================================================
\begin{document}

% 2段組設定とタイトル・著者情報
% \twocolumn[...] の中は1段組で表示されます
\twocolumn[%
	\presendate{1}{2025}{8}{8}  %% 発表日 (第1回/2025年/8月/8日)
    \papertitle{幾何音響理論に基づく高速鉄道騒音の\\リアルタイム可聴化システムの構築}
    \writer{r}{中央大学理工学部都市環境学科 学部4年\quad 増田 樹 \\ Tatsuki Masuda}
]

%**********************************************************
\section{はじめに}
騒音とは, 典型 7 公害の 1 つであり不快で好ましくない音の総称であり, 苦情件数は最多である.そのため,事前に騒音による影響を予測し,対策を講じることが極めて重要であり,近年ではコンピュータ技術の向上に伴い,波動音響理論や幾何響理論を用いた数値シミュレーションが広く行われている.

既往研究では,リアルタイムシミュレーションが可能な音響計算手法である幾何音響理論に着目し, 近年の高速鉄道による騒音問題に対して可視化及び可聴化をすることができる VR(Virtual Reality) 技術を用いた体験型鉄道騒音評価シミュレーションシステムの構築1) が行われてきた.

本研究では,地上走行では空力音が主な音の発生源となり,騒音問題が懸念されている電磁気浮上式高速鉄道に対して,VR 技術を用いた騒音評価システムの構築を目的とし,音の指向性として, 双指向性を考慮した電磁気浮上式高速鉄道騒音の伝播計算の結果と VR 空間内でのシミュレーション結果の比較, 音源数変更に伴うシミュレーション結果の比較を行った. またこれらのシミュレーションを行う際に重要となてくる実音源の定常音化についても学び, 作成したためそれらの音源の比較を行った.


%**********************************************************
\section{電磁気浮上式高速鉄道騒音評価システム}
\subsection{VR 環境}
本研究では,VR 装置として,図-1 に示す没入型 VR 装置 Holostage を用いる.この装置は,3 面(正面,側面, 底面)の大型スクリーンとそれぞれに対応した高性能プロジェクター,VR 空間内の利用者の動きを捉えるための, ワイヤレストラッキング装置及びそれらを制御する並列計算機から構成される. また,天井及び床には 12.1 チャンネルの音響機器を備えており,立体音響場を構築することが可能である.
本システムのフローチャートを図-2 に示す.入力データとしては,車両の走行条件,音源の音響パワーレベル, 構造物や軌道の幾何形状を設定する.また,時間ループ内において,車両の音源位置,VR 空間内の利用者(受音点)の位置情報をトラッキング装置より取得する.そして,それらの情報を用いて幾何音響理論に基づくモデル(ASJ RTN Model2018)2) により,利用者の位置情報における騒音レベルを計算する.

% % 図1のプレースホルダー
% \begin{figure}[h]
%     \centering
%     % \includegraphics[width=0.8\linewidth]{pic/holostage.png}
%     \caption{没入型 VR 装置 Holostage}
%     \label{fig:holostage}
% \end{figure}

% % 図2のプレースホルダー
% \begin{figure}[h]
%     \centering
%     % \includegraphics[width=0.9\linewidth]{pic/system_flow.png}
%     \caption{本システムの構成}
%     \label{fig:system_flow}
% \end{figure}

\subsection{可視化システム}
VR 空間の構築及び CG 映像の作成には CAVELib を用いており,プログラムは C++ を用いている.VR 空間内において刻々と変化する高速鉄道の位置情報を入力し, 観測者位置における音圧レベルを評価することが可能である.また,VR 空間の観測者(システム利用者)は,歩行およびコントローラを用いて VR 空間内を自由に移動可能であり,その座標位置はトラッキング装置により瞬時に捕捉されるため,リアルタイムに騒音を体験することができる.

\subsection{可聴化システム}
可聴化では,幾何音響理論に基づくモデルにより観測者位置における騒音レベルを計算し,その結果を音響プログラミングソフト Max を用いたプログラムによって音響計算結果に基づく立体音響信号を提示する.立体音響場の構築には,Ambisonics 手法4) を用いており,可視化と可聴化情報の共有は,OSC(Open Sound Control) プロトコルを用いた UDP/IP 通信によって行っている.

%**********************************************************
\section{適用例}
前章の計算手法を用いた浮上式高速鉄道騒音シミュレーションの妥当性の検証のため,実際に VR 空間上で騒音計を用いて測定した音圧レベルの値と,指向性モデルを用いた計算値を比較した.

\subsection{測定条件}
音源の音響パワーレベルを 120dB とした.受音点は音源から 20m 離れた位置とし,音源位置から 3m の高さとしている.なお,車両の CAD データは旧型の試験車両 MLX01 系を用いており,16 両編成としている.車両の速度は 500km/h とし,音源の位置は車両の連結部分の断面中心に設定した.(音源数 15)

\subsection{解析結果}
VR 空間上で騒音計を用いて測定した音圧レベルの値と,指向性モデルを用いた計算値を比較したグラフを図-3 に示す.車両編成を考慮した波形となっており,車両通過時は概ね形状が一致し,システムの妥当性を示すことができた.車両通過時の音圧レベルを比べると先頭車両通過時以外では概ね一致していることが確認できる. 車両通過前後では差異が生じているが,これは図-3 の車両通過前後に示すような,プロジェクターの動作音が主因となる暗騒音の影響によるものが考えられる.

% % 図3のプレースホルダー
% \begin{figure}[t]
%     \centering
%     % \includegraphics[width=0.9\linewidth]{pic/comparison.png}
%     \caption{測定値と計算値の比較}
%     \label{fig:comparison}
% \end{figure}

%**********************************************************
\section{定常音の作成練習}
今後の騒音測定に向け, 実音源から定常音への編集方法について学んだ. 定常音作成を行うことで, 各時間・空間ごとでの車両と観測者の位置座標により, 騒音の音の大きさを VR 上で再現可能になる. 質の良い定常音を作成することは CAVE でのシミュレーションのリアリティ向上に大きく影響してくる. そのため理解を深めるため, 練習もかねて既存の実音源を定常音化を行った. また既存の定常音もあるため, 自身で作成した定常音との比較を行った.

\subsection{作成手順}
まず走行時騒音の定常音になりうる音を抽出し, 複製を行う. 複製された音源を結合していく. ここで結合時に音源の繋がり部で不自然な音が発生しない用に修正を行う. 今回は図-4 に示す, 位相をずらした音源を複製し重ね合わせ定常音化する重ね合わせという修正と, 図-5 に示す複製された音源から順再生と逆再生を作成しそれらを交互に並べ定常音化するピンポンループという修正を行った. 今回使用する測定された実音源は去年測定を行った都留保守基地側で東京から名古屋方面へと走行するリニア新幹線の音源である. 受音点は音源から 5m 離れた位置とし, 人の身長を考慮して地上から 1.5m の高さとしている.なお 5 両編成とし, 車両の速度は 500km/h である. 今回は車両中心の 3 音源を抽出, 複製し作成した定常音, 中心の 1 音源を抽出, 複製した定常音, 中心より 1 つはやく発生した 1 音源を抽出, 複製し, 作成した定常音の三種類それぞれで重ね合わせなし, 二分の一波長位相重ね合わせあり, 三分の一波長位相重ね合わせあり, 五分の一波長位相重ね合わせあり, 重ね合わせなしピンポンループにおいて結合部が修正された定常音の作成を行った. その中でも既存の定常音に近しい音源では CAVE で実装し, その結果の解析, 比較を行った.

% % 図4, 5のプレースホルダー
% \begin{figure}[h]
%     \centering
%     % \includegraphics[width=0.9\linewidth]{pic/overlap.png}
%     \caption{重ね合わせの概要}
%     \label{fig:overlap}
% \end{figure}
% \begin{figure}[h]
%     \centering
%     % \includegraphics[width=0.9\linewidth]{pic/pingpong.png}
%     \caption{ピンポンループの概要}
%     \label{fig:pingpong}
% \end{figure}

\subsection{比較結果}
今回は全 15 種類の定常音を作成したが, 中には複製された音源の結合部がうまくいかず不自然な音が聴こえてしまう定常音も出来てしまった. そのため中でも違和感の少ない車両中心の 3 音源を抽出, 複製し作成した定常音と中心より 1 つはやく発生した 1 音源を抽出, 複製し, 作成した定常音の二つから重ね合わせなし, 二分の一波長位相重ね合わせあり, 重ね合わせなしピンポンループの結合部での修正を比較する. まず作成された定常音についての比較だが, 図-6, 図-7 に示すように既存の定常音源と波形が概ね一致していることがわかる. しかし CAVE 内で行った結果を比較すると図-8 卓越周波数であるはずの 6000 から 7000Hz での周波数領域でスペクトルレベルが卓越していないことがわかる. これは音源数が 1 つの場合でのほうが既存の定常音を用いた結果の波形に近いが図-9, どちらの結果のどの結合部の修正を行った場合でも波形が一致しているとはいえない結果となった.

% % 図6, 7などのプレースホルダー
% \begin{figure}[t]
%     \centering
%     % \includegraphics[width=0.9\linewidth]{pic/compare_3sources.png}
%     \caption{定常音化の比較 (音源数 3)}
%     \label{fig:compare_3sources}
% \end{figure}
% \begin{figure}[t]
%     \centering
%     % \includegraphics[width=0.9\linewidth]{pic/compare_1source.png}
%     \caption{定常音化の比較 (音源数 1)}
%     \label{fig:compare_1source}
% \end{figure}

%**********************************************************
\section{おわりに}
以下に結論をまとめる. 車両の通過時では計算値と測定値が概ね一致した.しかし,車両接近時と通過後では測定値との差が発生した.これはCAVE作動音などの暗騒音による影響, また室内での測定のため出力された音が残留している影響であると考える. またV R 空間内でのス

%**********************************************************
% 参考文献リスト
\begin{thebibliography}{9}
    % \bibitem{キー} 著者名: 論文名, 雑誌名, 巻, 号, pp.開始ページ-終了ページ, 発行年.
    \bibitem{ray_tracing} Kuttruff, H.: Room Acoustics, 5th ed., Spon Press, 2009.
    \bibitem{binaural} Blauert, J.: Spatial Hearing: The Psychophysics of Human Sound Localization, The MIT Press, 1997.
    \bibitem{train_noise} Thompson, D. J.: Railway Noise and Vibration: Mechanisms, Modelling and Means of Control, Elsevier, 2008.
\end{thebibliography}

\end{document}
